\section{Elementary Graph Algorithms}

\subsection{Representations of graphs}

\begin{description}

\descitem{22.1-1} \textit{Given an adjacency-list representation of a directed graph, how long does it take to compute the out-degree of every vertex? How long does it take to compute the in-degrees?}    

\begin{ex}
Soient $G = (V,E)$ un graphe non-orienté, $v \in V$ un de ses sommets et $\attrib{G}{Adj}$ sa représentation en liste adjacente. On peut calculer le degré sortant de $v$ en calculant simplement la longueur de $\attrib{G}{Adj}[v]$. Ainsi, le calcul de degré sortant de tous les sommets se fait en $\Theta(V + E)$.

Pour calculer le degré entrant de $v$ on doit parcourir $\attrib{G}{Adj}$ entièrement. De cette manière, il nous faudra $\Theta(VE)$ en temps. On peut aussi implémenter une liste de taille $|V|$, indexé par ses sommets dans lequel chaque élément est initialisé par 0. Dans ce cas là, il suffit de parcourir $\attrib{G}{Adj}$ une fois en comptant chaque occurrence des sommets. Cette méthode a $\Theta(V + E)$ en complexité de temps et $\Theta(V)$ en espace.
\end{ex}


\descitem{22.1-2} \textit{Give an adjacency-list representation for a complete binary tree on $7$ vertices. Give an equivalent adjacency-matrix representation. Assume that vertices are numbered from $1$ to $7$ as in a binary heap.}    

%TODO:Horizontally centering subfigure
\begin{ex}
\begin{figure}[H]
\begin{subfigure}{0.45\textwidth}
\begin{table}[H]
\centering
\begin{tabular}{lll}
1: & 2 & 3 \\
2: & 4 & 5 \\
3: & 6 & 7 \\
4: & 2 &   \\
5: & 2 &   \\
6: & 3 &   \\
7: & 3 &  
\end{tabular}
\end{table}
\caption{Liste adjacente}
\end{subfigure}
\begin{subfigure}{0.45\textwidth}
\begin{center}
$$
\begin{pmatrix}
  0 & 1 & 1 & 0 & 0 & 0 & 0\\
  0 & 0 & 0 & 1 & 1 & 0 & 0\\
  0 & 0 & 0 & 0 & 0 & 1 & 1\\
  0 & 1 & 0 & 0 & 0 & 0 & 0\\
  0 & 1 & 0 & 0 & 0 & 0 & 0\\
  0 & 0 & 1 & 0 & 0 & 0 & 0\\
  0 & 0 & 1 & 0 & 0 & 0 & 0\\
\end{pmatrix}
$$
\end{center}
\caption{Matrice adjacente}
\end{subfigure}
\caption{Représentation de l'arbre complet de 7 sommets en liste adjacente et matrice adjacente}
\end{figure}
\end{ex}
\descitem{22.1-3} \textit{The \textbf{transpose} of a directed graph $G = (V, E)$ is the graph $G^T = (V, E^T)$, where $E^T = \{(v,u) \in V \times V : (u,v) \in E \}$. Thus, $G^T$ is $G$ with all its edges reversed.  Describe efficient algorithms for computing $G^T$ from $G$, for both the adjacency-list and adjacency-matrix representations of $G$. Analyze the running times of your algorithms.}    
\begin{ex}
Pour transposer une liste adjacente, il est nécessaire de la parcourir entièrement, donc la complexité est de $\Theta(V + E)$.
\begin{codebox}
\Procname{\algo{Transpose-Adj-List}$(\id{Adj})$}
    \li Create a new adjacency-list \id{newAdj}
    \li \For each linked-list $i$ of \id{Adj}\Do
    \li \For each element $j$ of \id{Adj}[i] \Do 
    \li $\attrib{newAdj[j]}{append(i)}$ \End \End
\end{codebox}
Quant à une matrice adjacente, il suffit de parcourir les éléments  supérieurs et les échanger à leur élément symétrique. De cette manière, il y a au total $\frac{n(n-1)}{2}$ d'échanges, d'où la complexité $\Theta(V^2)$.
\begin{codebox}
\Procname{\algo{Transpose-Adj-Matrix}$(M)$}
    \li \For $i \gets 1$ \To $\attrib{M}{size}$ \Do
    \li \For $j \gets i+1$ \To $\attrib{M}{size}$ \Do
    \li $\func{swap}(M[i,j],M[j,i])$ \End \End
\end{codebox}
\end{ex}
%TODO:english quote
\descitem{22.1-4} \textit{Given an adjacency-list representation of a multigraph $G = (V, E)$, describe an $O(V + E)$-time algorithm to compute the adjacency-list representation of the \foreignquote{english}{equivalent} undirected graph $G' = (V,E')$, where $E'$ consists of the edges in $E$ with all multiple edges between two vertices replaced by a single edge and with all self-loops removed.}    
\begin{ex}
\begin{codebox}
\Procname{\algo{Multigraph-To-Undigraph}$(\id{Adj})$}
    \li Create an array $A$ having length $|V|$ 
    \li \For each linked-list \id{Adj}[i]\Do
    \li $\func{init}(A)$ \Comment initialize all element to \const{false}
    \li \For each element $j$ of \id{Adj}[i] \Do 
    \li $e = \id{Adj[i][j]}$
    \li \If $e == i$ \Comment remove loop \Then 
    \li \attrib{Adj[i]}{remove(j)} 
    \li \Else 
    \li \If $A[e] == \const{true}$ \Comment remove multiple edge \Then
    \li \attrib{Adj[i]}{remove(j)} 
    \li \Else 
    \li $A[e] \gets \const{true}$\End
    \End \End
\end{codebox}
\end{ex}
\descitem{22.1-5} \textit{The \textbf{square} of a directed graph $G = (V, E)$ is the graph $G^2 = (V,E^2)$ such that $(u,v) \in E^2$   if and only if $G$ contains a path with at most two edges between $u$ and $v$. Describe efficient algorithms for computing $G^2$ from $G$ for both the adjacency-list and adjacency-matrix representations of $G$. Analyze the running times of your algorithms.}    
\begin{ex}
Dans le cas de matrice adjacente, on peut savoir l'existence d'une arrête en temps constant. Pour vérifier qu'il existe une arrête $(u,w) \in G^2$, on cherche toute paire d'arrête $(u,v)$ avec $v$ le sommet intermédiaire, puis en cas d'existence de $(u,v_0) \in G$, on cherche toute paire d'arrête $(v_0,w) \in G$. D'une telle manière, l'algorithme prend $\Theta(E^3)$.
\begin{codebox}
\Procname{\algo{Square-of-Adj-Matrix}$(M)$}
\li Create a matrix $M_2$ which have the same dimension of $M$
\li \For each vertex $i$ in $M$ \Do
  \li \For each vertex $j$ in $M[i]$ \Do
    \li \If $M[i][j]==1$ \Then
      \li \For each element $k$ in $M[j]$ \Do
        \li \If $M[i][k] == 1$ \Then
          \li $M_2[i][k] \gets 1$ 
        \End
        \li $M_2[i][k] \gets 1$ 
      \End 
    \End 
  \End
\End
\end{codebox}
%TODO:add explanation
\begin{codebox}
\Procname{\algo{Square-of-Adj-List}$(\id{Adj})$}
\li Create a new adjacency-list \id{newAdj}
\li \For each linked-list $i$ of $\id{Adj}$ \Do 
    \li \For each element $j$ of $\id{Adj}[i]$ \Do
        \li $\attrib{newAdj[i]}{append}(j)$ \Comment Add 1-length path
        \li \For each element $k$ of $\id{Adj}[j]$ \Do
            \li $\attrib{newAdj[i]}{append}(k)$ \Comment Add 2-length path
        \End
    \End
\end{codebox}
\end{ex}
\descitem{22.1-6} \textit{Most graph algorithms that take an adjacency-matrix representation as input require time $\Omega(V^2)$, but there are some exceptions. Show how to determine whether a directed graph $G$ contains a \textbf{universal sink} -- a vertex with in-degree $|V| - 1$ and out-degree $0$ -- in time $O(V)$, given an adjacency matrix for $G$.}    

\begin{exrev}
On cherche, dans la matrice adjacente, une colonne qui contient $|V|-1|$ 1. De cette manière, il suffit de parcourir qu'une fois le graphe entier, d'où la complexité $O(V)$.

%TODO:write the algo
\begin{codebox}
\Procname{\algo{Square-of-Adj-List}$(\id{M})$}
\li \For each vertex in 
\end{codebox}
\end{exrev}
%TODO:refer TD
\descitem{22.1-7} \textit{The \textbf{incidence matrix} of a directed graph $G = (V,E)$ with no self-loops is a $|V| \times |E|$ matrix $B = b_{ij}$ such that}
\[
    b_{ij} = \left\{
    \begin{array}{ll}
        -1 & \text{if edge }$j$\text{ leaves vertex } $i$,\\
        1 & \text{if edge }$j$\text{ enters vertex } $i$,\\
        0 & \text{otherwise.}
    \end{array}
    \right.
\]
Describe what the entries of the matrix product $BB^T$ represent, where $B^T$ is the transpose of $B$.
%TODO:see hash table chapter
\descitem{22.1-8} \textit{Suppose that instead of a linked list, each array entry \id{Adj}[u] is a hash table containing the vertices $v$ for which $(u,v) \in E$. If all edge lookups are equally likely, what is the expected time to determine whether an edge is in the graph? What disadvantages does this scheme have? Suggest an alternate data structure for each edge list that solves these problems. Does your alternative have disadvantages compared to
the hash table?}    
\begin{exrev}

\end{exrev}

\end{description}

\subsection{Breadth-first search}
