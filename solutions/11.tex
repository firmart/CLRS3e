\section{Hash Tables}

\subsection{Direct-address tables}
\label{sub:direct_address_tables}

\begin{description}

\descitem{11.1-1} \textit{Suppose that a dynamic set $S$ is represented by a direct-address table $T$ of length $m$. Describe a procedure that finds the maximum element of $S$. What is the worst-case performance of your procedure ?}

\begin{ex}
  Recherche linéaire des éléments non-$\const{Nil}$, $O(m)$.
  %TODO:add procedure.
\end{ex}

\descitem{11.1-2} \textit{A \textbf{bit vector} is simply an array of bits (0s and 1s). A bit vector of length $m$ takes much less space than an array of $m$ pointers. Describe how to use a bit vector to represent a dynamic set of distinct elements with no satellites data. Dictionary operations should run in $O(1)$ time.)}

\begin{ex}

Supposons qu'on travaille sur des entiers, alors les procédures suivantes sont $O(1)$ en temps. La procédure \proc{Bit-Vector-Search} retourne \const{true} si le bit $k$ vaut $1$ et réciproquement.

\begin{codebox}
\Procname{\algo{Bit-Vector-Search}$(\id{b}, \id{k})$}
   \li \If $b[k] == 1$
\end{codebox}
\begin{codebox}
\Procname{\algo{Bit-Vector-Insert}$(\id{b}, \id{x})$}
   \li \Return $b[x] \gets 1$
\end{codebox}
\begin{codebox}
\Procname{\algo{Bit-Vector-Delete}$(\id{b}, \id{x})$}
   \li \Return $b[x] \gets 0$
\end{codebox}
\end{ex}

\descitem{11.1-3} \textit{}

\descitem{11.1-4} \textit{}


\end{description}


\subsection{Hash tables}
\label{sub:hash_tables}

\subsection{Hash functions}
\label{sub:hash_functions}

\subsection{Open addressing}
\label{sub:open_addressing}

\subsection{Perfect Hashing}
\label{sub:perfect_hashing}






