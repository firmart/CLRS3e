\section{Summation}

R\'ef\'erences : \cite{uic_hw1}

\subsection{Summation formulas and propreties}

\begin{description}
  \item [A.1-1] {\itshape Find a simple formula for $\sum_{k=1}^n(2k-1)$ }
   \begin{ex}
  \begin{align*}
    \sum_{k=1}^n(2k-1)  & = n(n+1)-n\\
    &= n^2.
  \end{align*}
\end{ex}
\item[A.1-2 $\star$]  {\itshape Show that $\sum_{k=1}^n1/(2k-1) = \ln (\sqrt{n}) + \O(1)$ by manipulating the harmonic series.}
    \begin{ex}
      \begin{align*}
        \sum_{k=1}^n \frac{1}{2k-1} &= H_n - \frac{1}{2}\sum_{k=1}^n\frac{1}{k}\\
       &= \frac{1}{2}H_n\\
       &= \ln(\sqrt{n}) + \O(1).
      \end{align*}
    \end{ex}
  \descitem{A.1-3} {\itshape Show that $\sum_{k=0}^{\infty} k^2x^k = x(1+x)/(1-x)^3$ for $ 0 < |x| < 1$.}
    \begin{ex}
    \begin{align*}
  \sum_{k=0}^\infty k^2x^k & = x \left( \frac{x}{(1-x)^2} \right)^\prime\\
      &= \frac{x}{(1-x)^2}+\frac{2x}{(1-x)^3}\\
      &= \frac{x(1+x)}{(1-x)^3}.
    \end{align*}
  \end{ex}
\item[A.1-4 $\star$] {\itshape Show that $\sum^\infty_{k=0}(k-1)/2^k = 0$.}
    \begin{ex}
    Soit $S=\sum_{k=0}^\infty\frac{k-1}{2^k}$. Alors $2S = \sum_{k=0}^{\infty}\frac{k-1}{2^{k-1}}= -2 + \sum_{k=0}^\infty\frac{k}{2^k}$. Donc 
    \begin{align*}
      2S-S &= -2 + \sum_{k=0}^\infty\frac{1}{2^k}\\
      &= 0.
     \end{align*}
   \end{ex}
 \item[A.1-5 $\star$] {\itshape Evaluate the sum $\sum_{k=1}^\infty(2k+1)x^{2k}$ for $|x| < 1$.}
   \begin{ex}
    \begin{align*}
      \sum_{k=1}^{\infty}(2k+1)x^{2k} &= \left(\sum_{k=1}^\infty x^{2k+1}\right)^\prime\\
    &= \left(\frac{x}{1-x^2} - x\right)^\prime\\
      &= \frac{3x^2}{1-x^2} + \frac{2x^4}{(1-x^2)^2}
    \end{align*}
  \end{ex}
  \descitem{A.1-6} {\itshape  Prove that $\sum_{k=1}^n\O(f_k(i)) = O\left(\sum_{k=1}^n f_k(i)\right)$ by using the linearity proprety of summations.}
    \begin{ex}
    Soit pour tout $k \in \llbracket 1, n \rrbracket, g_k \in \O(f_k)$, autrement dit $g_k \le c_kf_k$ est v\'erifi\'e \`a partir d'un rang $N_k$ avec $c_k > 0$. On a donc $\sum_{k=1}^n g_k \le \sum_{k=1}^ncf_k$ v\'erifi\'e \`a partir d'un rang $N = \max(N_1, \ldots,N_n)$ et $c = \max(c_1, \ldots,c_n)$. D'o\`u $\sum_{k=1}^n g_k \in \O(\sum_{k=1}^n f_k)$.
  \end{ex}
\descitem{A.1-7} {\itshape Evaluate the product $\prod_{k=1}^n2\cdot 4^k$.}
    \begin{ex}
  Soit $P=\prod_{k=1}^n2\cdot4^k = \prod_{k=1}^n2^{2k+1}$. On a 
  \begin{align*}
    \lg P &= \sum_{k=1}^n2k+1\\
    &= n(n+2)
  \end{align*}

  Finalement, $P = 2^{n(n+2)}$.
    \end{ex}
  \item[A.1-8 $\star$] {\itshape Evaluate the product $\prod_{k=2}^n(1-1/k^2)$.}
    \begin{ex}
  \begin{align*}
    \prod_{k=2}^n(1-\frac{1}{k^2}) &= \prod_{k=2}^n \frac{k-1}{k}\cdot\frac{k+1}{k}\\
      &= \frac{1}{2}\cdot\frac{n+1}{n}\\
      &= \frac{n+1}{2n}.
  \end{align*}
    \end{ex}
\end{description}

  \subsection{Bounding summations}

  \begin{description}
    \descitem{A.2-1} {\itshape Show that $\sum_{k=1}^n 1/k^2$ is bounded above by a constant.}
      \begin{ex}
      \begin{align*}
        \sum_{k=1}^n\frac{1}{k^2} &= 1 + \sum_{k=2}^n\frac{1}{k^2}\\
        &\le 1 + \int_1^n\frac{1}{x^2}\diff x\\
        &= 1 + (1-\frac{1}{n})\\
        &= 2 - \frac{1}{n}\\
        &\le 2.
      \end{align*}
    \end{ex}
    \descitem{A.2-2}{\itshape Find an asymptotic upper bound on the summation $\sum\limits_{k=0}^{\lfloor \lg n \rfloor}\lceil n/2^k \rceil$.}

    \descitem{A.2-3} {\itshape Show that the $n$th harmonic number is $\Omega(\lg n)$ by splitting the summation.}

    \descitem{A.2-4}{\itshape Approximate $\sum_{k=1}^nk^3$ with an integral.}
        \begin{ex}
      On a
      $$\int_0^nx^3\diff x \le \sum_{k=1}^nk^3 \le \int_1^{n+1}x^3\diff x$$ ce qui donne $$\frac{n^4}{4} \le \sum_{k=1}^nk^3 \le \frac{(n+1)^4-1}{4}.$$ Ainsi, $\sum_{k=1}^nk^3 = \Theta(n^4)$.
    \end{ex}

  \descitem{A.2-5} {\itshape Why didn’t we use the integral approximation (A.12) directly on $\sum_{k=1}^n1/k$ to obtain an upper bound on the {\itshape n}th harmonic number?}
    \begin{ex}
      Car la primitive de $1/x$ n'est pas d\'efinie en $0$.
    \end{ex}

  \end{description}

  \subsection{Problems}

  \begin{description}
    \descitem{A-1} {\itshape \bfseries Bounding summations}

        {\itshape Give asymptotically tight bounds on the following summations. Assume that $r > 0$ and $s > 0$ are constants.}

        \begin{Al}
          \item $\sum\limits_{k=1}^nk^r$.
          \item $\sum\limits_{k=1}^n\lg^sk$.
          \item $\sum\limits_{k=1}^nk^r\lg^sk$
        \end{Al}

        \begin{pbrev}
        \end{pbrev}

  \end{description}
